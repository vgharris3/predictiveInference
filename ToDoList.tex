\documentclass[12pt, a4paper]{article}
\usepackage[margin=0.5in]{geometry}

\usepackage{color}
\usepackage[dvipsnames]{xcolor}
\usepackage{hyperref}
\hypersetup{
    colorlinks=true,
    linkcolor=blue,
    urlcolor=blue,
    linktoc=all
}


\usepackage{amsmath}
\usepackage{mathtools}
\usepackage{amssymb}
\usepackage{cancel}
\usepackage{bm}
\usepackage{dsfont}
\usepackage{graphicx}
\usepackage{graphics}
\usepackage{xfrac}
\usepackage{array}
\setcounter{MaxMatrixCols}{40}

\usepackage{ulem} %just so I can strike through items on my todo list using \hl{text to be struck through}

\usepackage{enumerate}
\usepackage{enumitem}
\usepackage{multirow}

%inclusions carried over from past class homework formats
\usepackage{units}
\usepackage{fullpage}
\usepackage{alltt}
\usepackage{mathrsfs}
\usepackage{xcolor}
\usepackage{soul}

\usepackage{pgfplots}

\DeclarePairedDelimiter{\abs}{\lvert}{\rvert}
\newcommand*{\fontCourier}{\fontfamily{pcr}\selectfont}
\newcommand*\mean[1]{\overline{#1}}
\newcommand\scalemath[2]{\scalebox{#1}{\mbox{\ensuremath{\displaystyle #2}}}}

\setcounter{tocdepth}{5}
\setcounter{secnumdepth}{5}

% \usepackage{pdfpages}
\usepackage{Sweave}
\begin{document}
% \includepdf{TitlePage_MastersThesis}
% \includepdf{ThesisApprovalPage}
\Sconcordance{concordance:ToDoList.tex:ToDoList.Rnw:%
1 49 1 1 0 7 1 1 4 25 1}


% \tableofcontents
% \newpage



\title{To Do List}
\author{\Large Gabe Harris}
\maketitle

\begin{itemize}
  \item Read intros to listed texts (See Thesis section 2.1 Why Predictive Inference?) and summarize answers to that question in WhyPredictiveInference.Rnw
    \begin{itemize}
      \item Nate Silver's book (1st chapter or intro)
      \item \hl{Hoff's book}
      \item Geyser's book (1st chapter or intro)
      \item Aitchison and Dunsmore (1st chapter or intro)
      \item \hl{Dean's paper}
      \item maybe some googling
    \end{itemize}
  \item \hl{Create example comparing predictive inference result with plug in parameter result}
    \subitem \hl{Dean's suggestion:  1-sample binomial with small sample size.  E.g. 3 successes, 7 failures (Pr(success) < 0.5).  Difference will be more pronounced with smaller sample sizes.}
  \item No do not \hl{Combine rpredNormIG(), rpredNormIG2(), and rpredNormIGk() into one function}
  \item Read up on convergence in probability and justification for MCMC (Hoff, Casella \& Berger)
  \item Compare variance computations: var() function on MCMC sample vs direct computation from theory ($EX^2 - (EX)^2$)
  \item General guidance:  Read up on ``How to write a thesis in Latex," for example \url{https://www.overleaf.com/learn/latex/How_to_Write_a_Thesis_in_LaTeX_(Part_1)%3A_Basic_Structure}
  \item Figure formatting:  save figures as .png files and insert?  Or what?  Need better control over how figure appears in resulting pdf.
  \item \textcolor{red}{(0)} Incorporate any guidance from Dean.
  \item Write up explanation of how each function works
  \item Ask Melanie for ``Master's Thesis Style File"
\end{itemize}

\begin{itemize}
  \item Not unless Dean asks for it:  write intro paragraph (couple of sentences) for each model
  \item Exponential-Gamma random sampler:  draw a single theta from posterior or draw a new one for each prediction?  (Can't tell the difference from histograms)
  \item Beta-Binomial:  Use posterior of theta (which is a gamma) and then make prediction based on draw(s) of theta (like I'm doing for Exponential-Gamma?)  Why did I go to the trouble of using the inverse transform method before?  Try drawing $\theta\sim\text{Beta}(t+\alpha,N-t+\beta)$ and then predict t using \texttt{rbinom(S,N,theta)}.
  \item Establish S as random sample size, make sure consistent throughout
  \item Make sure consistent with $\tilde{y}$ for predicted value throughout
  \item add sample function calls to R Implementation sections (see end of BB right before example) \hl{done through Poisson-Gamma}
  \item add (model name) to example section headings
  \item Work through the math for the 2-sample Normal-Inverse Gamma full conditional distributions (see Hoff p. 128)
  \item Section 3 intro:  mention split of NormIG into 1, 2, and k sample models.
  \item Section 3 intro:  mention NormIG2() and NormIGk() only offer random sample functions.
  \item Establish S as random sample size throughout.
  \item In ``R Implementation" sections, put function parameters into function at beginning of intros
  \item Make sure consistent with use of $\tilde{y}$ for single prediction throughout.  Also mention this in intro.
  \item Modify R Implementation section for EG to describe c = censoring indicator vector
  \item Come up with real EG example
  \item Come up with real PG example
  \item check $\int_\theta$ for consistent use throughout
  \item Am I numbering all the formulas that need numbering?
  \item Review use of $\pi(\cdot)$ to denote prior distributions of parameters.  Right now my intro says I'm doing that but at least in the k-sample Normal-Inverse Gamma I'm not doing that.
  \item \hl{Review Normal Regression section and pare down for draft.}
  \item List of Figures
  \item List of Tables
  \item Citations guidance:  you should use the citation style appropriate for your discipline, following
the guidance of your committee.  Ask Melanie.
  \item Create appendix containing R code for functions and for examples.
  \item Check boxes for graduation
  \begin{itemize}
    \item schedule thesis defense (must be April 29 or earlier--Try for earlier--4/8 and 4/22 are off Fridays for me)
    \item change address at UAccess once we have closed on the house
    \item confirm/change info for graduation at \url{https://grad.arizona.edu/commencement} (see email from Guadalupe Estrella on January 31)
    \item Thesis submission:  create account here and submit thesis by deadline (May 13 so earlier) \url{http://www.etdadmin.com/arizona}
  \end{itemize}
\end{itemize}


{\huge Dean's Notes 1/4/2022}
\begin{enumerate}
  \item \hl{Remove the Chapter 1 and Chapter 2 from the chapter titles.}
  \item \hl{p 4, Abstract - You know that Bayesian inference need not be predictive.}
  \item p 4, section 2.1 \\
    I think you may need a citation for that first  sentence.\\
    \hl{I feel like you need several citation in this first paragraph.}\\
    \hl{How many of these assertions are your original ideas, and how many are borrowed?}\\

  \item \hl{p 5  What are the take-aways from this example?}
  \item \hl{p 8 Ch3  intro paragraph}\\
    \hl{Yes - add an intro paragraph}\\
    \hl{What about these problems makes them unique? or simple?}\\
    \hl{Also, describe (list) the problems you are going to address.}\\
  \item p9, bottom - The random sampling should work the same either way.  \textcolor{red}{Is one method preferable?}
  \item \hl{p10, bottom} \\
    \hl{The likelihood is a function of the parameter conditional on the data.}\\\\

    \hl{The conventional use of upper and lower case values for variables you know that Y is unobserved, y is observed.}\\\\

    \hl{Switching to the likelihood notation is a little less cumbersome.}  \\
    \hl{For discrete PMFs (like the binomial), the expression would be}\\
    \hl{Pr$(Y1 = y1,...,Yn=yn | \theta)$}\\\\

    \hl{For continuous pdfs its trickier.}

  \item \hl{p 11 - first para.  For the sentences above, I presume you want some censoring value and not a parameter theta?}
  \item \hl{p17 - section 3.3.3  I also don't know where you got this example. See note in red in theses text:  My data was made up (random) and the example is completely abstract}
  \item \hl{p21 3.4.1.3 - I don't think you need other values for kappa or nu.}
  \item \hl{p24. 3.4.2.3 - I would probably just say that Hoff provides the following example and we reproduce his description.}
  \item \hl{p27. The two-column format would be more standard, and would fit better with the tidy data format.  It would be easier to use.}
  \item \hl{p 33.  Comparing the values of beta-ols to their standard errors}\\
    \hl{This is the usual regression t-statistic for regression parameter estimates.}
  \item \hl{This is as far as I got.  dean}

\end{enumerate}

{\huge Questions for Dean 1/28/2022}
\begin{enumerate}
  \item \hl{NormIG2():  Subscript $N$ in the parameters $\mu_N,\gamma^2_N,\delta_N,\tau^2_N$. Interpret these as $N = N_1 + N_2$?  Sure.  Maybe add an explanation}
  \item annotating and numbering figures and tables?
  \item spacing?
  \item Poisson-Gamma example--use Hoff's? (see 3.3.3)
  \item Should I use "Dr." with  your name and others? (In intro, throughout, in bibliography) \textcolor{red}{yes--for my profs when first mentioned.  No for authors.}
  \item Scheduling defense
  \begin{itemize}
    \item Date, time, place
    \item Invite people (I'd rather not)
  \end{itemize}
\end{enumerate}









\end{document}
