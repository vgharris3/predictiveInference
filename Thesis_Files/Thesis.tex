\documentclass[12pt, a4paper]{article}
\usepackage[margin=0.5in]{geometry}

\usepackage{color}
\usepackage[dvipsnames]{xcolor}
\usepackage{hyperref}
\hypersetup{
    colorlinks=true,
    linkcolor=blue,
    urlcolor=blue,
    linktoc=all
}


\usepackage{amsmath}
\usepackage{mathtools}
\usepackage{amssymb}
\usepackage{cancel}
\usepackage{bm}
\usepackage{dsfont}
\usepackage{graphicx}
\usepackage{graphics}
\usepackage{xfrac}
\usepackage{array}
\setcounter{MaxMatrixCols}{40}

\usepackage{enumerate}
\usepackage{enumitem}
\usepackage{multirow}

%inclusions carried over from past class homework formats
\usepackage{units}
\usepackage{fullpage}
\usepackage{alltt}
\usepackage{mathrsfs}
\usepackage{xcolor}
\usepackage{soul}

\usepackage{pgfplots}

\DeclarePairedDelimiter{\abs}{\lvert}{\rvert}
\newcommand*{\fontCourier}{\fontfamily{pcr}\selectfont}
\newcommand*\mean[1]{\overline{#1}}
\newcommand\scalemath[2]{\scalebox{#1}{\mbox{\ensuremath{\displaystyle #2}}}}

\setcounter{tocdepth}{5}
\setcounter{secnumdepth}{5}

\usepackage{pdfpages}
\usepackage{Sweave}
\begin{document}
\includepdf{TitlePage_MastersThesis}
\includepdf{ThesisApprovalPage}
\Sconcordance{concordance:Thesis.tex:Thesis.Rnw:%
1 49 1 1 0 7 1 1 4 99 1 1 29 1 12 69 1 1 45 1 2 156 1 1 26 1 2 39 1 1 %
21 1 4 27 1}


\tableofcontents
\newpage


%%%%%%%%%%%%%%%%%%%%%%%%%%%%%%%%%
%%INTRODUCTION
%%%%%%%%%%%%%%%%%%%%%%%%%%%%%%%%%
\section{Thesis Abstract}
  \begin{itemize}
    \item (paragraph) Statement of the thesis topic and objectives
    \item (paragraph) Explanation of R package
  \end{itemize}

\clearpage

\section{Introduction:  Predictive Inference}

  \subsection{Why is predictive inference important?}

  \subsection{Difference between parametric inference and predictive inference}

    \subsubsection{When is predictive inference more useful?}

    \subsubsection{When is parametric inference more useful?}
      [examples, comparisons]

  \subsection{The Bayesian Parametric Prediction Format}
    [Geisser p. 49]\\

        Let $$f\left(x^{(N)},x_{(M)}|\theta\right) = f\left(x_{(M)}|x^{(N)},\theta\right)f\left(x^{(N)}|\theta\right).$$

    Here $x^{(N)}$ represents observed events and $x_{(M)}$ are future events.  We calculate

    $$f\left(x_{(M)},x^{(N)}\right) = \int f\left(x^{(N)},x_{(M)}|\theta\right)p(\theta)d\theta$$

    where $p(\theta)$ is the prior density and

    $$f\left(x_{(M)}|x^{(N)}\right) = \frac{f\left(x_{(M)},x^{(N)}\right)}{f\left(x^{(N)}\right)} = \int f\left(x_{(M)}|\theta\right)p\left(\theta|x^{(N)}\right)d\theta$$

    where

    $$p\left(\theta|x^{(N)}\right) \propto f\left(x^{(N)}|\theta\right)p(\theta).$$

  \subsection{[Maybe] Example of Difference between results from Plug-in estimator and results using Predictive Inference}

\clearpage

\section{Chapter 1:  Predictive Problems with Conjugate Priors}

  [Problems with closed-form solutions.  These problems will be what the R package is designed for.  Use problems from Geisser, Casella \& Berger (Bayesian chapter), other sources.  Regression problem--predictive distributions of models that include and exclude some predictor]

  \subsection{Prediction of Future Successes:  Beta-Binomial (Geisser p. 73)}


    \subsubsection{Derivation}

      Let $X_i$ be independent binary variables with Pr$(X_i = 1) = \theta$, and let $T = \sum X_i$.  Then $T$ has probability

      $${N\choose t}\theta^t(1-\theta)^{N-t}.$$

      \vspace{5mm}

      Assume $\theta\sim\text{Beta}(\alpha,\beta)$, so

      \vspace{5mm}

      $$p(\theta) = \frac{\Gamma(\alpha + \beta)\theta^{\alpha - 1}(1 - \theta)^{\beta - 1}}{\Gamma(\alpha)\Gamma(\beta)}.$$

      \vspace{5mm}

      Then

      \vspace{5mm}

      $$p\left(\theta|X^{(N)}\right) = \frac{\Gamma(N+\alpha+\beta)\theta^{t+\alpha-1}(1-\theta)^{N-t+\beta-1}}{\Gamma(t+\alpha)\Gamma(N-t+\beta)}$$

      \vspace{5mm}

      \noindent So for $R = \sum_{i=1}^M X_{N+i}$ we have Beta-Binomial predictive distribution

\begin{flalign*}
  \text{Pr}[R=r|t]
  &= \int {M\choose r}\theta^r(1-\theta)^{M-r}p\left(\theta|X^{(N)}\right)d\theta\\
  &\\
  &= {M\choose r}\int \theta^r(1-\theta)^{M-r}\frac{\Gamma(N+\alpha+\beta)}{\Gamma(t+\alpha)\Gamma(N-t+\beta)}\theta^{t+\alpha-1}(1-\theta)^{N-t+\beta-1}d\theta\\
  &\\
  &= \frac{M!}{r!(M-r)!}\frac{\Gamma(N+\alpha+\beta)}{\Gamma(t+\alpha)\Gamma(N-t+\beta)}\int\theta^{r+t+\alpha-1}(1-\theta)^{M-r+N-t+\beta-1}d\theta\\
  &\\
  &= \frac{\Gamma(M+1)\Gamma(N+\alpha+\beta)\Gamma(r+t+\alpha)\Gamma(M-r+N-t+\beta)}{\Gamma(r+1)\Gamma(M-r+1)\Gamma(t+\alpha)\Gamma(N-t+\beta)\Gamma(M+N+\alpha+\beta)}
\end{flalign*}

\clearpage

    \subsubsection{R Implementation}

This result has been used to create ``standard" R functions dpredBB(), ppredBB(), and rpredBB() for the Beta-Binomial distribtuion for density, cumulative probability, and random sampling, respectively (see appendix).  These functions are exercised in the following example.


    \subsubsection{Example}

Suppose $t=5$ successes have been observed out of $N=10$ binary events, $\alpha = 2$ and $\beta = 8$.  For $M = 1000$ future observations, the figures below show the predictive distribution from dpredBB(), the cumulative distribution from ppredBB(), and a histogram of random draws from rpredBB().


\includegraphics{Thesis-002}



    \subsection{Survival Time:  Exponential-Gamma (Geisser p. 74)}


    \subsubsection{Derivation}

      Suppose $X^{(N)} = \left(X^{(d)},X^{(N-d)}\right)$ where $X^{(d)}$ represents copies fully observed from an exponential survival time density
          $$f(x|\theta) = \theta e^{-\theta x}$$
      and $X^{(N-d)}$ represents copies censored at $x_{d+1},...,x_N$, respectively.  Hence
          $$L(\theta)\propto\theta^d e^{-\theta N\bar{x}}$$
      when $N\bar{x} = \sum_1^N{x_i}$, as shown below.\\

      The usual exponential likelihood is used for the fully observed copies, whereas for the censored copies we need Pr$(x > \theta) = 1 - \text{Pr}(x\leq\theta) = 1 - F(x|\theta) = 1 - (1 - e^{-\theta x}) = e^{-\theta x}$.  Thus the overall likelihood is

      $$L(\theta|x) = \prod_{i=1}^d\theta e^{-\theta x_i}\prod_{i=d+1}^N e^{-\theta x_i} = \theta^d e^{-\theta N\bar{x}}$$

      Assuming a Gamma$(\delta,\gamma)$ prior for $\theta$,

       $$p(\theta) = \frac{\gamma^\delta\theta^{\delta - 1}e^{-\gamma\theta}}{\Gamma(\delta)}$$

       we obtain the posterior

       %$$p\left(\theta|X^{(N)}\right) = \frac{p\left(x^{(N)}|\theta\right)p(\theta)}{\int p\left(X^{(N)}|\theta\right)p(\theta)d\theta} = \frac{(\gamma+N\bar{x})^{d+\delta}\theta^{d+\delta - 1}e^{-\theta(\gamma+N\bar{x})}}{\Gamma(d+\delta)}$$

        \begin{flalign*}
          p\left(\theta|X^{(N)}\right)
          &= \frac{p\left(x^{(N)}|\theta\right)p(\theta)}{\int p\left(X^{(N)}|\theta\right)p(\theta)d\theta}\\
          &\\
          &= \frac{\theta^d e^{-\theta N\bar{x}}\cdot\frac{\gamma^\delta\theta^{\delta - 1}e^{-\gamma\theta}}{\Gamma(\delta)}}{\int\left(\theta^d e^{-\theta N\bar{x}}\cdot\frac{\gamma^\delta\theta^{\delta - 1}e^{-\gamma\theta}}{\Gamma(\delta)}\right)d\theta}\\
          &\\
          &= \frac{\cancel{\frac{\gamma^\delta}{\Gamma(\delta)}}\left(\theta^{d+\delta - 1}e^{-\theta(\gamma+N\bar{x})}\right)}{\cancel{\frac{\gamma^\delta}{\Gamma(\delta)}}\int\left(\theta^{d+\delta - 1}e^{-\theta(\gamma+N\bar{x})}\right)d\theta}\\
          &\\
          &= \frac{\frac{(\gamma+N\bar{x})^{d+\delta}}{\Gamma(d+\delta)}\left(\theta^{d+\delta - 1}e^{-\theta(\gamma+N\bar{x})}\right)}{\cancel{\frac{(\gamma+N\bar{x})^{d+\delta}}{\Gamma(d+\delta)}\int\left(\theta^{d+\delta - 1}e^{-\theta(\gamma+N\bar{x})}\right)d\theta}}\\
          &\\
          &= \frac{(\gamma+N\bar{x})^{d+\delta}\theta^{d+\delta - 1}e^{-\theta(\gamma+N\bar{x})}}{\Gamma(d+\delta)}
        \end{flalign*}

    with the Gamma$(d+\delta,\gamma+N\bar{x})$ density in the next to last step integrating to $1$.\\

    Thus the survival time predictive probability is

    \begin{flalign*}
      P\left(X = x|\theta,X^{(N)}\right)
      &= \int p\left(\theta|X^{(N)}\right)p(x|\theta)d\theta\\
      &\\
      &= \int\frac{(\gamma+N\bar{x})^{d+\delta}\theta^{d+\delta - 1}e^{-\theta(\gamma+N\bar{x})}}{\Gamma(d+\delta)}\cdot\theta e^{-\theta x}d\theta\\
      &\\
      &= (d+\delta)(\gamma+N\bar{x})^{d+\delta}\int\frac{\theta^{(d+\delta + 1) - 1}e^{-\theta(\gamma+N\bar{x} + x)}}{(d+\delta)\Gamma(d+\delta)}d\theta\\
      &\\
      &= \frac{(d+\delta)(\gamma+N\bar{x})^{d+\delta}}{\left(\gamma+N\bar{x}+x\right)^{d+\delta+1}}\int\frac{\left(\gamma+N\bar{x}+x\right)^{d+\delta+1}\theta^{(d+\delta + 1) - 1}e^{-\theta(\gamma+N\bar{x} + x)}}{\Gamma(d+\delta+1)}d\theta\\
      &\\
      &= \frac{(d+\delta)(\gamma+N\bar{x})^{d+\delta}}{\left(\gamma+N\bar{x}+x\right)^{d+\delta+1}}
    \end{flalign*}

    (simplifying by constructing a Gamma$(d+\delta+1,\gamma+N\bar{x}+x)$ density in the final integrand.)\\



    \subsubsection{R Implementation}

This result has been used to create standard format R functions dpredEG(), ppredEG(), and rpredEG() for the Gamma-Exponential distribtuion for density, cumulative probability, and random sampling, respectively (see appendix).  These functions are exercised in the following example.


    \subsubsection{Example}

Suppose $d=800$ out of $N = 1000$ copies have been observed, and the remaining $200$ censored.  Say $\delta = 20$, $\gamma=5$, and we are interested in the number of survivors out of $M = 1000$ future observations.  The figures below illustrate the predictive probability using dpredEG() and rpredEG(), along with a histogram of a random sample taken using rpredEG().


\includegraphics{Thesis-003}

\clearpage

  \subsection{Poisson-Gamma Model (Hoff p. 43ff)}
    \subsubsection{Derivation}
    [using Hoff's notation and variable names below.  Should I convert this to Geisser's $x^{(N)},x_{(M)}$ convention for uniformity throughout my thesis?]\\\\
      Suppose $Y_1,...,Y_n|\theta\overset{i.i.d.}{\sim}\text{Poisson}(\theta)$ with Gamma prior $\theta\sim\text{Gamma}(\alpha,\beta)$.  That is,

      \begin{flalign*}
        P\left(Y_1 = y_1,...,Y_n = y_n|\theta\right)
        &= \prod_{i=1}^n p\left(y_i|\theta\right)\\
        &\\
        &= \prod_{i=1}^n\frac{1}{y!}\theta^{y_i}e^{-\theta}\\
        &\\
        &= \left(\prod_{i=1}^n\frac{1}{y!}\right)\theta^{\sum y_i}e^{-n\theta}\\
        &\\
        &= c\left(y_1,...,y_n\right)\theta^{\sum y_i}e^{-n\theta}
      \end{flalign*}

      and

      $$p(\theta) = \dfrac{\beta^\alpha}{\Gamma(\alpha)}\theta^{\alpha-1}e^{-\beta\theta}, \theta, \alpha, \beta > 0.$$

\bigskip

      Then we have posterior distribution

      \begin{flalign*}
        p\left(\theta|y_1,...,y_n\right)
        &= \dfrac{p\left(y_1,...,y_n|\theta\right)p(\theta)}{\int_\theta p\left(y_1,...,y_n|\theta\right)p(\theta)}\\
        &\\
        &= \dfrac{p\left(y_1,...,y_n|\theta\right)p(\theta)}{p\left(y_1,...,y_n\right)}\\
        &\\
        &= \dfrac{1}{p\left(y_1,...,y_n\right)}\theta^{\sum y_i}e^{-n\theta}\dfrac{\beta^\alpha}{\Gamma(\alpha)}\theta^{\alpha - 1}e^{-\beta\theta}\\
        &\\
        &= C\left(y_1,...,y_n,\alpha,\beta\right)\theta^{\alpha+\sum y_i - 1}e^{-(\beta + n)\theta}\\
        &\\
        &\sim \text{Gamma}\left(\alpha+\sum y_i,\beta + n\right).
      \end{flalign*}


      Here

      \begin{flalign*}
        C\left(y_1,...,y_n,\alpha,\beta\right)
        &= \dfrac{1}{p\left(y_1,...,y_n\right)}\cdot\dfrac{\beta^\alpha}{\Gamma(\alpha)}\\
        &\\
        &= \dfrac{1}{\int_\theta p\left(y_1,...,y_n|\theta\right)p(\theta)}\cdot\dfrac{\beta^\alpha}{\Gamma(\alpha)}\\
        &\\
        &= \dfrac{1}{\int_\theta\left(\prod\frac{1}{y_i!}\right)\theta^{\sum y_i}e^{-n\theta}\cancel{\left(\frac{\beta^\alpha}{\Gamma(\alpha)}\right)}\theta^{\alpha-1}e^{-\beta\theta}}\cdot\cancel{\left(\frac{\beta^\alpha}{\Gamma(\alpha)}\right)}
        &\\
        &= \dfrac{1}{\left(\prod\frac{1}{y_i!}\right)\frac{\Gamma(\alpha + \sum y_i)}{(\beta+n)^{\alpha+\sum y_i}}\int_\theta \frac{(\beta+n)^{\alpha+\sum y_i}}{\Gamma(\alpha+\sum y_i)}\theta^{\sum y_i+\alpha-1}e^{-(\beta+n)\theta}}\\
        &\\
        &= \dfrac{\prod_{i=1}^n y_i!(\beta+n)^{\alpha+\sum y_i}}{\Gamma(\alpha+\sum y_i)}
      \end{flalign*}

      Call this constant $C_n$ (for $n$ observations).

\bigskip

      Note that an additional observation $y_{n+1} = \tilde{y}$ the constant becomes

      $$C_{n+1} = \dfrac{\prod_{i=1}^{n+1} y_i!(\beta+n+1)^{\alpha+\sum_{i=1}^{n+1} y_i}}{\Gamma(\alpha+\sum_{i=1}^{n+1} y_i)}.$$

      Also note that the marginal joint distribution of $k$ observations is

      $$p\left(\tilde{y}|y_1,...,y_k\right) = \dfrac{1}{C_k}\dfrac{\beta^\alpha}{\Gamma(\alpha)}.$$

      For future observation $\tilde{y}$, then, we compute predictive distribution

      \begin{flalign*}
        p\left(\tilde{y}|y_1,...,y_n\right)
        &= \dfrac{p\left(y_1,...,y_n,\tilde{y}\right)}{p\left(y_1,...,y_n\right)} = \dfrac{p\left(y_1,...,y_{n+1}\right)}{p\left(y_1,...,y_n\right)}
        = \dfrac{\frac{1}{C_{n+1}}\cancel{\frac{\beta^\alpha}{\Gamma(\alpha)}}}{\frac{1}{C_n}\cancel{\frac{\beta^\alpha}{\Gamma(\alpha)}}}
        = \dfrac{C_n}{C_{n+1}}\\
        &\\
        &= \dfrac{\dfrac{\prod_{i=1}^n y_i!(\beta+n)^{\alpha+\sum_{i=1}^n y_i}}{\Gamma(\alpha+\sum_{i=1}^n y_i)}}{\dfrac{\prod_{i=1}^{n+1} y_i!(\beta+n+1)^{\alpha+\sum_{i=1}^{n+1} y_i}}{\Gamma(\alpha+\sum_{i=1}^{n+1} y_i)}}\\
        &\\
        &= \dfrac{\Gamma\left(\alpha+\sum_{i=1}^{n+1}y_i\right)(\beta+n)^{\alpha+\sum_{i=1}^n y_i}}{\left(y_{n+1}!\right)\Gamma\left(\alpha+\sum_{i=1}^n y_i\right)(\beta+n+1)^{\alpha+\sum_{i=1}^{n+1}y_i}}\\
        &\\
        &= \dfrac{\Gamma\left(\alpha+\sum_{i=1}^n y_i + \tilde{y}\right)(\beta+n)^{\alpha+\sum_{i=1}^n y_i}}{\left(\tilde{y}!\right)\Gamma\left(\alpha+\sum_{i=1}^n y_i\right)(\beta+n+1)^{\alpha+\sum_{i=1}^n y_i + \tilde{y}}}\\
        &\\
        &= \dfrac{\Gamma\left(\alpha+\sum y_i+\tilde{y}\right)}{\Gamma(\tilde{y}+1)\Gamma(\alpha+\sum y_i)}\cdot \left(\dfrac{\beta+n}{\beta+n+1}\right)^{\alpha+\sum y_i} \cdot \left(\dfrac{1}{\beta+n+1}\right)^{\tilde{y}}\\
      \end{flalign*}

This is a negative binomial distribution:  $\tilde{y}\sim NB\left(\alpha+\sum y_i,\beta+n\right)$, for which

\begin{flalign*}
  E\left[\tilde{Y}|y_1,...,y_n\right] &= \dfrac{a+\sum{y_i}}{b+n} = E\left[\theta|y_1,...,y_n\right];\\
  &\\
  \text{Var}\left[\tilde{Y}|y_1,...,y_n\right] &= \dfrac{a+\sum{y_i}}{b+n}\dfrac{b+n+1}{b+n}\\
  &\\
  &=\text{Var}\left[\theta|y_1,...,y_n\right]\times(b+n+1)\\
  &\\
  &=E\left[\theta|y_1,...,y_n\right]\times\dfrac{b+n+1}{b+n}\\
\end{flalign*}

\vspace{5mm}

\hrule

\vspace{5mm}

[Showing here that it is indeed a NB distribution]

$$\theta\sim NB(\alpha,\beta)\Rightarrow p(\theta) = \binom{\theta+\alpha-1}{\alpha - 1}\left(\dfrac{\beta}{\beta+1}\right)^\alpha\left(\dfrac{1}{\beta+1}\right)^\theta$$

\begin{center}so\end{center}

\begin{flalign*}
  \tilde{y}\sim NB\left(\alpha + \sum{y_i}),\beta+n\right)\Rightarrow p(\tilde{y})
  &= \binom{\tilde{y}+\alpha+\sum{y_i}-1}{\alpha+\sum{y_i}-1}\left(\dfrac{\beta+n}{\beta+n+1}\right)^{\alpha+\sum{y_i}}\left(\dfrac{1}{\beta+n+1}\right)^{\tilde{y}}\\
  &\\
  &= \dfrac{\left(\alpha + \sum{y_i} + \tilde{y} - 1\right)!}{\left(\alpha + \sum{y_i} - 1\right)!\left(\tilde{y}\right)!}\left(\dfrac{\beta+n}{\beta+n+1}\right)^{\alpha+\sum{y_i}}\left(\dfrac{1}{\beta+n+1}\right)^{\tilde{y}}\\
  &\\
  &= \dfrac{\Gamma\left(\alpha + \sum{y_i} + \tilde{y}\right)}{\Gamma\left(\alpha + \sum{y_i}\right)\Gamma\left(\tilde{y}+1\right)}\left(\dfrac{\beta+n}{\beta+n+1}\right)^{\alpha+\sum{y_i}}\left(\dfrac{1}{\beta+n+1}\right)^{\tilde{y}}
\end{flalign*}

\vspace{5mm}

\hrule

\vspace{5mm}

      \bigskip



      [This is the result in Hoff.  The straightforward derivation below is off by a constant multiple.  Need to figure out what went awry.]



      \begin{flalign*}
        p\left(\tilde{y}|y_1,...,y_n\right)
        &= \int_0^\infty p\left(\tilde{y}|\theta,y_1,...,y_n\right)p\left(\theta|y_1,...,y_n\right)d\theta\\
        &\\
        &= \int p\left(\tilde{y}|\theta\right)p\left(\theta|y_1,...,y_n\right)d\theta\\
        &\\
        &= C\int\left(\dfrac{1}{\tilde{y}!}\theta^{\tilde{y}}e^{-\theta}\right)\theta^{\alpha+\sum y_i - 1}e^{-(\beta+n)\theta}d\theta\\
        &\\
        &= \dfrac{C}{\tilde{y}!}\int\theta^{\tilde{y}+\alpha+\sum y_i - 1}e^{-(\beta+n+1)\theta}d\theta\\
        &\\
        &= \dfrac{C\Gamma\left(\tilde{y}+\alpha+\sum y_i\right)}{\Gamma\left(\tilde{y}+1\right)(\beta+n+1)^{\tilde{y}+\alpha+\sum y_i}}\int\dfrac{(\beta+n+1)^{\tilde{y}+\alpha+\sum y_i}}{\Gamma\left(\tilde{y}+\alpha+\sum y_i\right)}\theta^{\tilde{y}+\alpha+\sum y_i - 1}e^{-(\beta+n+1)\theta)}d\theta\\
        &\\
        &= C\cdot\dfrac{\Gamma\left(\tilde{y}+\alpha+\sum y_i\right)}{\Gamma\left(\tilde{y}+1\right)(\beta+n+1)^{\tilde{y}+\alpha+\sum y_i}}\\
        &\\
        &= \dfrac{\prod_{i=1}^n y_i!(\beta+n)^{\alpha+\sum y_i}}{\Gamma(\alpha+\sum y_i)}\cdot\dfrac{\Gamma\left(\tilde{y}+\alpha+\sum y_i\right)}{\Gamma\left(\tilde{y}+1\right)(\beta+n+1)^{\tilde{y}+\alpha+\sum y_i}}\\
        &\\
        &= \prod_{i=1}^n y_i! \cdot \dfrac{\Gamma\left(\tilde{y}+\alpha+\sum y_i\right)}{\Gamma(\tilde{y}+1)\Gamma(\alpha+\sum y_i)}\cdot \left(\dfrac{\beta+n}{\beta+n+1}\right)^{\alpha+\sum y_i} \cdot \left(\dfrac{1}{\beta+n+1}\right)^{\tilde{y}}\\
      \end{flalign*}

\textcolor{red}{Hoff p.47:
  \begin{itemize}
    \item $b$ is interpreted as the number of prior observations
    \item $a$ is interpreted as the sum of counts from $b$ prior observations
  \end{itemize}
}

\textcolor{red}{  Hoff p. 49 (Birth rate example):  $a = 2, b = 1$. }


    \subsubsection{R Implementation}

This result has been used to create standard format R functions dpredPG(), ppredPG(), and rpredPG() for the Poisson-Gamma distribution for density, cumulative probability, and random sampling, respectively (see appendix).  These functions are exercised in the following example.\\

    Developing the random sample function rpredPG():  I need to establish the support of the predictive distribution $f_x$ from which to sample.  the uniroot() function is not working because it keeps feeding non-integer values to dnbinom().  Strategy: a modified bisection method as follows:\\

    \begin{enumerate}
      \item set a desired tolerance $\epsilon$.
      \item Find the expected value $E_x$ (closed formula, see above).
      \item Step to the right of $E_x$ by whole integers, in the sequence $E_x + \{1,2,4,...2^n\}$, stopping at $U=f_x\left(E_x + 2^n\right) < 0$.  This is the upper bound for the bisection method.
      \item Bisect the interval, rounding to the nearest integer.  Call the resulting mid-interval number $B$.
      \item If B is positive, test whether $0 \leq f_x(B) \leq \epsilon$.  If so, DONE.  If not:
      \item Establish new interval, choosing endpoints from $E_x$, $B$, and $U$ so that the interval straddles $0$, and repeat the steps until the condition in step 5 is reached.
    \end{enumerate}

    \subsubsection{Example}

Suppose we have 10 prior observations with counts 27, 79, 21, 100, 8, 4, 37, 15, 3, 97.  Let $\alpha = 11$ and $\beta = 3$.  For $\tilde{y} = 1:100$ possible future occurrences, the figures below show the predictive distribution from dpredPG(), the cumulative distribution from ppredPG(), and a histogram of random draws from rpredPG().

\includegraphics{Thesis-004}

  \subsection{Normal Observation with Normal-Inverse Gamma Prior}
    \subsubsection{One sample}
      \paragraph{Derivation}
      [Hoff p. 69ff]\\
        Let $\left\{Y_1,...,Y_n|\theta,\sigma^2\right\}\overset{i.i.d.}{\sim}N\left(\theta,\sigma^2\right)$.  Then the joint sampling density is

        \begin{flalign*}
          p\left(y_1,...,y_n|\theta,\sigma^2\right)
          &= \prod_{i=1}^n p\left(y_i|\theta,\sigma^2\right)\\
          &\\
          &= \prod_{i=1}^n \dfrac{1}{\sqrt{2\pi\sigma^2}}e^{-\frac{1}{2}\left(\frac{y_i - \theta}{\sigma}\right)^2}\\
          &\\
          &= \left(2\pi\sigma^2\right)^{-\sfrac{n}{2}}e^{-\frac{1}{2}\sum_{i=1}^n\left(\frac{y_i - \theta}{\sigma}\right)^2}.\\
        \end{flalign*}

        It can be shown that $\left\{\sum{y_i^2},\sum{y_i}\right\}$ and hence $\left\{\bar{y},s^2\right\}$ are sufficient statistics, where $\bar{y} = \sum{y_i}/n$ and $s^2 = \sum\left(y_i - \bar{y}\right)^2/(n-1)$.\\


        \vdots

        Following Hoff (p. 74ff), for joint inference on both $\theta$ and $\sigma$, assume priors

        \begin{flalign*}
          \frac{1}{\sigma^2} &\sim \text{gamma}\left(\sfrac{\nu_0}{2},\sfrac{\nu_0\sigma_0^2}{2}\right)\\
          &\\
          \theta|\sigma^2 &\sim \text{normal}\left(\mu_0,\sfrac{\sigma^2}{\kappa_0}\right)\\
        \end{flalign*}

        where $\left(\sigma_0^2,\nu_0\right)$ are the sample variance and sample size of prior observations, and $\left(\mu_o, \kappa_0\right)$ are the sample mean and sample size of prior observations.\\

        \textcolor{red}{Are there different sets of prior observations for the two different prior distributions?  I.e. does $\nu_0 = \kappa_0$?}\\

        Note:  $\mu_0$, $\kappa_0$, $\nu_0$, and $\sigma_0^2$ come from prior knowledge. [in the Hoff example (Midge Wing Length), $\kappa_0$ and $\nu_0$ are both set to $1$ so that "our prior distributions are only weakly centered around these estimates from other populations."]\\

        From this we derive joint posterior

        \begin{flalign*}
          \left\{\theta|y_1,...,y_n,\sigma^2\right\} &\sim \text{normal}\left(\mu_n,\sfrac{\sigma^2}{\kappa_n}\right)\\
          &\\
          \left\{\sigma^2|y_1,...,y_n\right\} &\sim \text{inverse-gamma}\left(\sfrac{\nu_n}{2},\sfrac{\sigma^2_n\nu_n}{2}\right).
        \end{flalign*}

        where

        \begin{flalign*}
          \kappa_n &= \kappa_0 + n\\
          &\\
          \mu_n &= \frac{\kappa_0\mu_0+n\bar{y}}{\kappa_n}\\
          &\\
          \nu_n &= \nu_0 + n\\
          &\\
          \sigma_n^2 &= \frac{1}{\nu_n}\left[\nu_0\sigma_0^2 + (n-1)s^2 + \frac{\kappa_0 n}{\kappa_n}\left(\bar{y}-\mu_0\right)^2\right].\\
        \end{flalign*}

        Here $\bar{y} = \frac{1}{n}\sum_{i=1}^n y_i$ is the sample mean and $s^2 = \frac{1}{n-1}\sum_{i=1}^n\left(y_i - \bar{y}\right)^2$ is the sample variance.\\

        From the joint posterior distribution we generate marginal samples by means of the Monte Carlo method (Hoff, p. 77):

        \begin{flalign*}
          \begin{matrix}
            \sigma^{2(1)}\sim \text{inverse-gamma}\left(\nu_n/2,\sigma^2_n\nu_n/2\right), & \theta^{(1)}\sim \text{normal}\left(\mu_n,\sigma^{2(1)}/\kappa_n\right) \\
            \vdots  & \vdots  \\
            \sigma^{2(S)}\sim \text{inverse-gamma}\left(\nu_n/2,\sigma^2_n\nu_n/2\right), & \theta^{(S)}\sim \text{normal}\left(\mu_n,\sigma^{2(S)}/\kappa_n\right) \\
          \end{matrix}
        \end{flalign*}

        For prediction of future $\tilde{y}|y_1,...,y_n,\theta,\sigma^2$, generate $\tilde{y}_i \sim \text{normal}\left(\theta^{(i)},\sigma^{2(i)}\right)$.\\


      \paragraph{R Implementation}
      Standard format R functions dpredNormIG(), ppredNormIG(), and rpredNormIG() have been created for the Normal-Inverse Gamma distribution for density, cumulative probability, and random sampling, respectively (see appendix).  For the random sampler rpredNormIG(), the Monte-Carlo method described above was directly employed.  The predictive density and cumulative density functions depend on the random sample, and utilize Kernel Density Estimation (KDE) and R's built-in density() function.  The KDE is computed by definition, using a normal kernel:

      $$\hat{f}_K(x) = \frac{1}{n}\sum_{i=1}^n\frac{1}{h}K\left(\frac{x-X_i}{h}\right),$$

      where

      \begin{flalign*}
        X_i & \text{ is the random sample generated using rpredNormIG()}\\
        &\\
        K & \text{ is Normal(0,1)}\\
        &\\
        h & \text{ is the bandwidth from R's density() function (that is, } h = \text{ density}(X_i)\text{\$bw)}\\
      \end{flalign*}



      These functions are exercised in the following example.\\

        \clearpage

        \textit{Example (Hoff p. 72ff, using data from Grogan and Wirth (1981)):  Midge wing length}\\

        Grogan and Wirth (1981) provide 9 measurements of midge wing length, in millimeters:  $y = \{1.64, 1.7, 1.72, 1.74, 1.82, 1.82, 1.82, 1.90, 2.08\}$. Prior studies suggest values $\mu_0 = 1.9$ and $\sigma_0^2 = 0.01$.  We choose $\kappa_0 = \nu_0 = 1$ ``...so that our prior distributions are only weakly centered around these estimates from other populations" (Hoff p. 76).  We compute

        \begin{flalign*}
          \bar{y} &= 1.804\\
          &\\
          \text{var}(y) &= 0.0169\\
          &\\
          \kappa_n &= 1 + 9 = 10\\
          &\\
          \mu_n &= \frac{1 \cdot 1.9 + 9 \cdot 1.804}{10} = 1.814\\
          &\\
          \nu_n &= 1 + 9 = 10\\
          &\\
          \sigma_n^2 &= \frac{1}{10}\left[1 \cdot 0.01 + (9-1) \cdot 0.0169 + \frac{1 \cdot 9}{10}\left(1.804 - 1\right)^2\right] = 0.0153\\
        \end{flalign*}

        Thus $\sfrac{\nu_n}{2} = 5$ and $\sfrac{nu_n\sigma_n^2}{2} = 0.7662$ and we have posteriors

        \begin{flalign*}
          \left\{\theta|y_1,...,y_n,\sigma^2\right\} &\sim \text{normal}\left(1.814,\sfrac{\sigma^2}{10}\right)\\
          &\\
          \left\{\sigma^2|y_1,...,y_n\right\} &\sim \text{inverse-gamma}(5,0.7662)\\
        \end{flalign*}

\begin{Schunk}
\begin{Soutput}
           [,1]
kn    10.000000
mun    1.814000
sig2n  0.015324
nun   10.000000
\end{Soutput}
\begin{Soutput}
33.74 sec elapsed
\end{Soutput}
\begin{Soutput}
1.68 sec elapsed
\end{Soutput}
\end{Schunk}
\includegraphics{Thesis-005}



        \clearpage

        Skipping ahead

        \vdots

        posteriors given Jeffrey's Prior:

        $$\left\{1/\sigma^2|y_1,...,y_n\right\}\sim \text{gamma}\left(\frac{n}{2},\frac{n}{2}\frac{1}{n}\sum\left(y_i - \bar{y}\right)^2\right)$$

        same as

        $$\left\{1/\sigma^2|y_1,...,y_n\right\}\sim \text{gamma}\left(\frac{n}{2},\frac{1}{2}\sum\left(y_i - \bar{y}\right)^2\right)$$



        $$\left\{\theta|\sigma^2, y_1,...,y_n\right\}\sim \text{normal}\left(\bar{y},\frac{\sigma^2}{n}\right)$$

        \textcolor{red}{dig into Bedrick notes and homework for Jeffrey's prior}

\includegraphics{Thesis-006}


      \paragraph{Example}
    \subsubsection{Two samples}
      \paragraph{Derivation}
      \paragraph{R Implementation}
      \paragraph{Example}
    \subsubsection{$k$ samples}
      \paragraph{Derivation}
      \paragraph{R Implementation}
      \paragraph{Example}
      \paragraph{Ranking Treatments}

\clearpage


\section{Chapter 2:  Normal Regression with Zellner's $g$-prior}
  \paragraph{Derivation}
  \paragraph{R Implementation}
  \paragraph{Example}


\clearpage

\section{Conclusion}

\end{document}
